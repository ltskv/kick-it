\section{Video Recording from the Nao Camera}

For the purposes of debugging and also for the final presentation, we wanted to
record what the robot sees during the program execution. NAOqi SDK provides a
function to write the camera video to a file, but has a limitation of allowing
the capture from only one camera at a time, which was not optimal for us. We
overcame this limitation, by exploiting the fact, that the NAOqi SDK didn't
impose any restrictions on reading individual frames from the cameras into the
memory. So, during the test runs we started a separate thread, where the camera
frames from both cameras were read into memory one by one, and after the robot
has completed the execution of his task, the recorded frame sequences were
written to video files with the help of OpenCV. This approach has a downside,
that the frames can only be read at irregular and unpredictable intervals, so
the framerate of the resulting video couldn't be calculated, which means that
the playback speed of the videos needed to be adjusted afterwards using video
editing programs. Furthermore, due to computational resource limitations of the
Nao, the frames could have been captured only in low resolution. However, the
quality of the resulting videos was sufficient for successful debugging and
also for the presentation.
