\section{Strategy Overview}

\begin{figure}[ht]
  \includegraphics[width=\textwidth]{\fig striker-flowchart}
  \caption{Overview of the goal scoring strategy}
  \label{p figure strategy-overview}
\end{figure}

Now that all of the milestones are completed, we will present a short overview
of the whole goal scoring strategy, the block diagram of which can be found in
the figure \ref{p figure strategy-overview}. At the very beginning the robot
will detect the ball and turn to ball, as described in the section \ref{j sec
  turning to ball}. After that, the distance to the ball will be calculated,
the goal will be detected, and the direction to goal will be determined. If the
ball is far away \textit{and} the ball and the goal are strongly misaligned,
then the robot will try to approach the ball from the appropriate side,
otherwise the robot will approach the ball directly. These approach steps will
be repeated until the robot is close enough to the ball to start aligning to
the goal, but in the practice one step of approach from the side followed by a
short direct approach should suffice. When the ball is close, the robot will
check if it is between the goalposts, and will perform necessary adjustments if
that's not the case. After the ball and the goal are aligned, the robot will
align its foot with the ball and kick the ball. For now, we assumed that the
ball will reach the goal and so the robot can finish execution.
