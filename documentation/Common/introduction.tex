\chapter{Introduction}

RoboCup \cite{robocup} is an international competition in the field of
robotics, the ultimate goal of which is to win a game of soccer against a human
team by the middle of the 21st century. The motivation behind this objective is
the following: It is impossible to achieve such an ambitious goal with the
current state of technology, which means that the RoboCup competitions will
drive scientific and technological advancement in such areas as computer
vision, mechatronics and multi-agent cooperation in complex dynamic
environments. The RoboCup teams compete in five different leagues: Humanoid,
Standard Platform, Medium Size, Small Size and Simulation. Our work in this
semester was based on the rules of the \textit{Standard Platform League}. In
this league all teams use the same robot \textit{Nao}, which is being produced
by SoftBank Robotics. We will describe the capabilities of this robot in
more detail in the next chapter.

A couple of words need to be said about the state-of-the-art. One of the most
notable teams in the Standard Platform League is \textit{B-Human}
\cite{bhuman}. This team represents the University of Bremen and in the last
nine years they won the international RoboCup competition six times and twice
were the runner-up. The source code of the framework that B-Human use for
programming their robots is available on GitHub, together with an extensive
documentation, which makes the B-Human framework a frequent starting point for
RoboCup beginners.

\section{Our Objective and Motivation}
\label{sec problem statement}

In this report we are going to introduce the robotics project, which our team
worked on during the Summer Semester 2018. The main objective of our work was
to explore a possible strategy for fast goal scoring. There are three main
aspects of our motivation behind this objective. The first one is the fact,
that goal scoring is crucial for winning soccer games, therefore fast and
effective goal scoring will bring the team closer to victory. Secondly, in
order to score a goal, many problems and tasks need to be solved, which we will
describe in close detail in the following chapters. The work on these tasks
would allow us to acquire new competences, which we could then use to
complement the RoboCup team of the TUM. Finally, this objective encompasses
many disciplines, such as object detection, mechatronics or path planning,
which means that working on it might give us a chance to contribute to the
research in these areas.

Having said that, we hope that our project will be a positive contribution to
the work being done at the Institute for Cognitive Systems and that this
report will help future students to get familiar with our results and continue
our work.
