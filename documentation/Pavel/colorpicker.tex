\section{Color Calibration}

All our detection algorithms require color calibration, and when the lighting
conditions on the field change, colors might have to be recalibrated. For us
this meant that a tool was necessary, that could simplify this process as far
as possible. For this reason, we implemented a small OpenCV-based program, that
we called \verb|Colorpicker|. This program can access various video sources, as
well as use still images for calibration. The main interface contains the
sliders for adjusting the HSV interval, as well as the video area,
demonstrating the resulting binary mask. The colors can be calibrated for three
targets: ball, goal and field; and the quality of detection, depending on the
chosen target is demonstrated in the tool's video area. When the program is
closed, the calibration values are automatically saved to the settings file
\verb|nao_defaults.json|. The interface of the Colorpicker is demonstrated in
the figure \ref{p figure colorpicker}.

\begin{figure}[ht]
  \includegraphics[width=\textwidth]{\fig colorpicker}
  \caption{Interface of the Colorpicker}
  \label{p figure colorpicker}
\end{figure}
