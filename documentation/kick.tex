\section{Kick}

The final milestone in the goal scoring project is naturally the kick. Before
we started working on the kick, we formulated some requirements, which our
implementation must satisfy. Firstly and most importantly, the robot shouldn't
fall down when performing the kick. Secondly, the kick must have the sufficient
strength, so that ideally only one kick is necessary for the ball to reach the
goal. Therefore, due to time constraints we implemented the simplest possible
kick, that would satisfy those requirements.

The procedure is as follows. First the robot will use its ankle joints to shift
its weight to the base leg. After this, the robots will be able to lift the
kicking leg for the swing. Finally, the robot will perform the swing and return
to the standing position. Both raising the leg and doing the swing require
precise coordinated joint movements, so we had to conduct experiments to
establish the correct joint angles and the movement speed.

An important drawback of our implementation is that the swing makes the whole
process slower, but we weren't able to design a strong and stable kick without
using the swing. Nevertheless, the tests that we performed have shown, that our
implementation satisfies our requirements, and hence the last milestone was
successfully completed.

% \begin{figure}[ht]
  % \includegraphics[width=\textwidth]{\fig kick}
  % \caption{Kick sequence}
  % \label{p figure kick}
% \end{figure}
