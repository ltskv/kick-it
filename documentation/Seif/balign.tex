\section{Ball Alignment}

Now that the ball and the goal are aligned, the robot has to move into a
position, from which the kick can be performed. Depending on the situation, it
may be feasible to select the foot, with which the kick should be performed,
but due to time constraints we programmed the robot to kick with the left foot
only. So, the task now is to place the ball in front of the left foot. We
realized, that when the ball is in the correct position, then its image in the
lower camera should be within a certain region. We experimentally determined
the extents of this region, which is schematically presented in figure \ref{p
  figure ball-alignment}. The algorithm therefore is for the robot to gradually
adjust its position in small steps, until the ball image reaches the target,
after which the robot will proceed with the kick. Our tests have shown, that
this method while being relatively simple, works sufficiently robust, which
means that we didn't have the situations, when the robot missed the ball after
alignment or even hit the ball with an edge of the foot.

\begin{figure}[ht]
  \includegraphics[width=\textwidth]{\fig ball-align}
  \caption{Ball alignment}
  \label{p figure ball-alignment}
\end{figure}
